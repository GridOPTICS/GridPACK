\section{Environment}\label{environment}

GridPACK applications need to initialize several libraries in order to execute
properly. These can be initialized explicitly by the user in their
application but they can also be initialized by creating a single
\texttt{\textbf{Environment}} object at the start of the code. This module uses
just the \texttt{\textbf{gridpack}} namespace. The constructor
for this object will automatically call all the appropriate initialization
functions for libraries used by GridPACK. This object can also be used to
support an inline help message that can be used to document how to use the
application.

There are two main constructors for this class

{
\color{red}
\begin{Verbatim}[fontseries=b]
Environment(int argc, char **argv)
\end{Verbatim}
}

{
\color{red}
\begin{Verbatim}[fontseries=b]
Environment(int argc, char **argv, const char* help)
\end{Verbatim}
}

The \texttt{\textbf{argc}} and \texttt{\textbf{argv}} arguments are the standard
command line variables used in C and C++ \texttt{\textbf{main}} programs. These
will be passed to the math library and MPI initialization. Other options can also be
passed from the command line as well. Currently, the only command line
options that are supported directly by the \texttt{\textbf{Environment}} class
are \texttt{\textbf{-h}} and \texttt{\textbf{-help}}. If the application is invoked
with these options, then the program will print out whatever information is stored
in the \texttt{\textbf{help}} variable and then exit.

When using the \texttt{\textbf{Environment}} object to initialize GridPACK, it
is important to set the main body of the code apart from the creation of the
\texttt{\textbf{Environment}} object. The code should have a structure that
looks like

{
\color{red}
\begin{Verbatim}[fontseries=b]
int main(int argc, char **argv)
{
  gridpack::Environment env(argc,argv);
  {
    // Main body of code goes here
  }
  return 0;
}
\end{Verbatim}
}

This form guarantees that all objects created by the application are destroy
first before the \texttt{\textbf{Environment}} destructor is called. This is
important since otherwise the destructors for some objects in the application
may be called after the Global Array or MPI environments have been shut down and
if they rely on calls in the Global Array or MPI libraries, the application will
not exit cleanly.
