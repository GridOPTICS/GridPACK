\documentclass[12pt]{article}
\usepackage{color}
\renewcommand{\baselinestretch}{1.0}
\begin{document}
\newcommand{\tmsym}{^{\mbox{TM}}}
\begin{titlepage}
\begin{center}
{\LARGE DRAFT: Network Module in GridPACK$\tmsym$}
\end{center}
\end{titlepage}
%\tableofcontents
\newpage
\pagestyle{plain}
\section{Introduction}
This module is designed to create a network, partition it across multiple
processors, and allow applications to add and modify fields to the different
buses and branches. The core functionality is located in the BaseNetwork class
which is a templated class that allows developers to specify arbitrary classes
for the buses and branches in the network. Only a single bus or branch object is
associated with each bus or branch, so all possible bus and branch behaviors
must be incorporated into each bus or branch object. See the section on network
components for more information about bus and branch objects. This document will
first present information on BaseNetwork methods that are likely to be used by
application developers. Additional functionality in the BaseNetwork class that
is available but is unlikely to be used outside the GridPACK$\tmsym$ framework
itself is presented at the end of this documentation.

The BaseNetwork class is nested within the \texttt{gridpack} and \texttt{network}
namespaces. Within the GridPACK$\tmsym$ framework itself, all BaseNetwork
methods are prefixed with \texttt{gridpack::network::BaseNetwork} and we encourage
developers to use complete namespaces as well. Developers may elect to use the
\texttt{using} statement instead. Any file creating or using a BaseNetwork object
should include the header file
\begin{verbatim}
#include "gridpack/network/base_network.hpp"
\end{verbatim}
As long as the correct location of the \texttt{gridpack} link has been specified
in the application makefile, all other GridPACK$\tmsym$ include files will
automatically be included.  The BaseNetwork class can be subclassed to create
networks tailored to specific applications, if desired. The BaseNetwork class
itself is templated and is declared as
\begin{verbatim}

template <class _bus, class _branch> class BaseNetwork

\end{verbatim}
where \texttt{\_bus} and \texttt{\_branch} are component classes representing
the behaviors of buses and branches, respectively.

\section{Initializing and terminating the network module}
A BaseNetwork object can be create and destroyed using simple constructors and
destructors. The constructor takes a \texttt{parallel::Communicator} object as
an argument. This contains the MPI communicator as well as other processor group
information and is designed to support the creation of network objects on subsets
of the processor world group.
\color{blue}
\begin{verbatim}
explicit void BaseNetwork(
    const gridpack::parallel::Communicator &comm)
comm (IN): communicator on which network is defined
\end{verbatim}
\normalcolor
This function creates a BaseNetwork object. The BaseNetwork class is a templated
class, so actual creation of a base network oject has the form
\begin{verbatim}

gridpack::network::BaseNetwork<BusClass,BranchClass>
   network(comm);

\end{verbatim}
Once created, the network class contains no buses and branches until these are
explicitly added. This is usually done by invoking other modules in
GridPACK$\tmsym$, but it can also be done by adding them using functions described
below.

\color{blue}
\begin{verbatim}
void ~BaseNetwork(void)
\end{verbatim}
\normalcolor
The destructor for the BaseNetwork class cleans up all memory used by the
BaseNetwork object.

\section{Local and global network topology information}
These functions provide information about the total number of buses and branches
in the entire network as well as the number of buses and branches contained
locally on the processor. They also provide information on whether a bus or
branch is ``owned'' by a processor or whether it represents a ghost image of a
bus or branch owned by another processor.

\color{blue}
\begin{verbatim}
int totalBuses(void)
\end{verbatim}
\normalcolor
This function returns the number of buses in the entire network. This is a
global operation so it must be called on all processors in the network. If this
number is needed repeatedly, it should be evaluated once and then cached
locally.

\color{blue}
\begin{verbatim}
bool getActiveBus(int idx)
idx (IN): local index of bus
\end{verbatim}
\normalcolor
This function can be used to inquire whether a bus is local to the process or is a
ghost bus. It returns true if the bus is local and false otherwise.

\color{blue}
\begin{verbatim}
int numBuses(void)
\end{verbatim}
\normalcolor
This function returns the number of buses on the calling processor. This
includes both local and ghost buses so the sum of the number of buses obtained
using the \texttt{numBuses} operation across all processors will exceed the
number of buses obtained using the \texttt{totalBuses} call.


\color{blue}
\begin{verbatim}
boost::shared_ptr<_bus> getBus(int idx)
idx (IN): local index of bus
\end{verbatim}
\normalcolor
This function returns a \texttt{shared\_ptr} to the bus object indexed by the
local index \texttt{idx}.

\color{blue}
\begin{verbatim}
int totalBranches(void)
\end{verbatim}
\normalcolor
This function returns the number of branches in the entire network. This is a
global operation so it must be called on all processors in the network. If this
number is needed repeatedly, it should be evaluated once and then cached
locally.

\color{blue}
\begin{verbatim}
int numBranches(void)
\end{verbatim}
\normalcolor
This function returns the number of branches on the calling processor. This
includes both local and ghost branches so the sum of the number of branches obtained
using the \texttt{numBranches} operation across all processors will exceed the
number of branches obtained using the \texttt{totalBranches} call.

\color{blue}
\begin{verbatim}
bool getActiveBranch(int idx)
idx (IN): local index of branch
\end{verbatim}
\normalcolor
This function can be used to inquire whether a branch is local to the process or is a
ghost branch. It returns true if the branch is local and false otherwise.

\color{blue}
\begin{verbatim}
boost::shared_ptr<_branch> getBranch(int idx)
idx (IN): local index of branch
\end{verbatim}
\normalcolor
This function returns a \texttt{shared\_ptr} to the branch object indexed by the
local index \texttt{idx}.
\end{document}
