\documentclass[12pt]{article}
\renewcommand{\baselinestretch}{1.0}
\begin{document}
\begin{titlepage}
\begin{center}
{\LARGE DRAFT: Interface Specification for PGRID Module}
\end{center}
\end{titlepage}
%\tableofcontents
\newpage
\pagestyle{plain}
\section{Introduction}
This module is designed to create a network, partition it across multiple
processors, and allow applications to add and modify fields to the different
nodes and links.
\section{Initialization and Termination}
This section will desribe basic initialization and termination functions for the
PGRID module.
\subsection{PGRID\_Init}
\begin{verbatim}
void PGRID_Init(void)
\end{verbatim}
This function initializes the PGRID module. It must be called before any other
functions in the PGRID module can be called and must be called on all
processors.
\subsection{PGRID\_Set\_group}
\begin{verbatim}
void PGRID_Set_group(int nproc,
                     int *list)
\end{verbatim}
\begin{itemize}
\item (IN) \texttt{nproc}: number of processors in group
\item (IN) \texttt{list}: list of processors in group
\end{itemize}
This function can be used to subdivide the entire set of processors into
subgroups. Each subgroup will support a separate grid instance. This function
must be called on all processors. Not sure if this is really a good way to
handle this.
\subsection{PGRID\_Add\_link}
\begin{verbatim}
void PGRID_Add_link(int inode1, int inode2)
\end{verbatim}
\begin{itemize}
\item (IN) \texttt{inode1}: global index of node 1 at end of link
\item (IN) \texttt{inode2}: global index of node 2 at end of link
\end{itemize}
This function adds a link to the network.  The link is defined by the global
indices \texttt{inode1} and \texttt{inode2} that define the two nodes at the
end of the link.
\subsection{PGRID\_Add\_node}
\begin{verbatim}
void PGRID_Add_node(int node_id)
\end{verbatim}
\begin{itemize}
\item (IN) \texttt{node\_id}: global index of node
\end{itemize}
This function adds a node to the network. The global index \texttt{node\_id}
must represent a unique integer index for the node but does not need to be part
of a sequence of consecutive numbers. The partioning function will derive a
unique integer index that represents an internal ordering of the nodes.
\subsection{PGRID\_Partition}
\begin{verbatim}
void PGRID_Partition(void)
\end{verbatim}
This function partitions the grid so that it is distributed evenly across all
processors. After calling this, it is no longer possible to add nodes and links
to the grid.
\subsection{PGRID\_Terminate}
\begin{verbatim}
void PGRID_Terminate(void)
\end{verbatim}
This function closes down the PGRID module and cleans up all utility data
structures. No PGRID module functions can be called after this routine is
called. This function must be called on all processors.

\subsection{Accessing local grid information}
Routines for accessing local number of nodes and links.
\begin{itemize}
\item \begin{verbatim}
int PGRID_Get_prop(char *prop_name)
\end{verbatim}
\item \begin{verbatim}
int PGRID_Get_link_nodes(int local_link_index,
                         int node1,
                         int node2)
\end{verbatim}
\item \begin{verbatim}
int PGRID_Get_node_link(int local_node_index,
                        int num_links,
                        int *link_list)
\end{verbatim}
\end{itemize}

\subsection{Adding (and deleting?) node and link fields. Accessing link field
elements}
\begin{itemize}
\item \begin{verbatim}
void PGRID_Add_field(char *field_type,
                     char *data_type,
                     char *field_name,
                     int *field_handle)
\end{verbatim}
\end{itemize}
\subsection{Accessing field properties}
Accessing or modifing properties of individual elements
\begin{itemize}
\item \begin{verbatim}
void PGRID_Get_elem(int elem_index,
                    int field_handle,
                    void *value)
\end{verbatim}
\item \begin{verbatim}
void PGRID_Set_elem(int elem_index,
                    int field_handle,
                    void *value)
\end{verbatim}
\end{itemize}
\end{document}
