\documentclass[12pt]{article}
\usepackage{amssymb,amsmath}
\begin{document}
\title{Weighted-Least-Square(WLS) State Estimation}
\author{Yousu Chen \\PNNL}
\maketitle
This document is a description of how to formulate the weighted-least squares
(WLS) state estimation problem. Most of the formulation is based on the
book by Abur and Exposito\footnote{Ali Abur, Antonio Gomez Exposito, ``Power
System State Estimation Theroy and Implementation'', CRC Press }.

Power system state estimation is a central component in power system Energy
Management Systems. A state estimator receives field measurement data from
remote terminal units through data transmission systems, such as a Supervisory
Control and Data Acquisition (SCADA) system. Based on a set of non-linear
equations relating the measurements and power system states (i.e.  bus voltage,
and phase angle), a state estimator fine-tunes power system state variables by
minimizing the sum of the residual squares.  This is the well-known WLS method.

The mathematical formulation of the WLS state estimation algorithm for an
$n$-bus power system with $m$ measurements is given below.

\subsection*{Basic Equations}
The starting equation for the WLS state estimation algorithm is
\begin{equation}\label{eq:main}
z=\begin{bmatrix} z_1\\ z_2\\ .\\ .\\ .\\ z_m\\ \end{bmatrix} =
\begin{bmatrix} h_1(x_1,x_2,...,x_n)\\ h_2(x_1,x_2,...,x_n)\\ .\\ .\\ .\\ h_m(x_1,x_2,...,x_n)\\ \end{bmatrix}+\begin{bmatrix} e_1\\ e_2\\ .\\ .\\ .\\ e_m\\ \end{bmatrix}
=h(x)+e
\end{equation} 
The vector $z$ of $m$ measured values is
\[
z^{T} = \begin{bmatrix} z_1 & z_2 && ... & z_{m}\end{bmatrix}
\]
The vector $h$
\[
h^{T} = \begin{bmatrix} h_1(x) & h_2(x) && ... & h_{m}(x) \end{bmatrix}
\]
containing the non-linear functions $h_{i}($x$)$ relates the predicted value of
measurement $i$ to the state vector $x$ containing $n$ variables
\[
x^{T} = \begin{bmatrix} x_{1} & x_{2} & ... & x_{n} \end{bmatrix}
\]
and $e$ is the vector of measurement errors
\[
e^{T} = \begin{bmatrix} e_{1} & e_{2} & ... & e_{m} \end{bmatrix}
\]

The measurement errors $e_i$ are assumed to satisfy the following statistical
properties. First, the errors have zero mean
\begin{equation}
E(e_i) = 0, i = 1,...,m
\end{equation} 
Second, the errors are assumed to be independent,
($E[e_i e_j]=0$ for $i\ne j$), such that the covariance matrix is diagonal
\begin{equation}
Cov(e)=E(e\cdot e^T) = R =diag \{{\sigma_1 ^2,\sigma_2^2,...,\sigma_m ^2}\}
\end{equation} 

The objective function is then given by the relations
\begin{equation}\label{eq:JX}
J(x)=\sum_{i=1}^{m}(z_i-h_i(x))^2/R_{ii} = [z-h(x)]^TR^{-1}[z-h(x)] 
\end{equation} 
The minimization condition is
\begin{equation}\label{eq:gx}
g(x)=\frac{\partial J(x)}{\partial x} = -H^T(x)R^{-1}[z-h(x)] = 0 
\end{equation} 
where $H(x)=\partial h(x)/\partial x$. Expanding $g(x)$ into its Taylor series
leads to the expression
\begin{equation}\label{eq:Taylor}
g(x)= g(x^k) + G(x^k)(x-x^k) + ... = 0
\end{equation} 
where the $k+1$ iterate is related to the $k$th iterate via
\[
x^{k+1} = x^k - G(x^k)^{-1}g(x^k)
\]
and $G(x^k)$ is the gain matrix
\[
G(x^k) = \frac{\partial g(x^k)}{\partial x} = H^T(x^k)R^{-1}H(x^k)
\]
Note the each iterate $g(x^k)$ still satisfies
\[
g(x^k) = -H^T(x^k)R^{-1}(z-h(x^k))
\]

\subsection*{The Normal Equation and solution procedure:}
The normal equation for the state estimation calculation follows from
equation (\ref{eq:Taylor}) and is given by the expression
\begin{equation}\label{eq:normal}
G(x^k)\Delta x^{k+1} = H^T(x^k)R^{-1}(z-h(x^k)) \\*
\end{equation} 
where $\Delta x^{k+1} = x^{k+1}-x^k$. The WLS SE algorithm is based on this
equation and consists of the following steps
\begin{enumerate}
\item Set $k=0$
\item Initialize the state vector $x^{k}$, typical a flat start
\item Calculate the measurement function $h(x^k)$
\item Build the measurement Jacobian $H(x^k)$
\item Calculate the gain matrix of $G(x^k)= H^T(x^k)R^{-1}H(x^k)$
\item Calculate the RHS of the normal equation $H^T(x^k)R^{-1}(z-h(x^k))$
\item Solve the normal equation \ref{eq:normal} for $\Delta x^k$
\item Check for convergence using $\max \left |  \Delta x^k \right |\le \epsilon $
\item If not converged, update $x^{k+1} = x^k+\Delta x^k$ and go to 3. Otherwise
stop
\end{enumerate}


\subsection*{The measurement function $h(x^k)$}

The measured quantities and their relation to the state variables are listed
below
\begin{enumerate}
\item Real and reactive power injection at bus $i$
\begin{equation}\label{eq:Pi}
P_i = V_i \sum_{i\ne j}V_j(G_{ij}\cos \theta _{ij}+B_{ij}\sin \theta _{ij})
\end{equation} 
\begin{equation}\label{eq:Qi}
Q_i = V_i \sum_{i\ne j}V_j(G_{ij}\sin \theta _{ij}+B_{ij}\cos \theta _{ij})
\end{equation} 
\item Real and reactive power flow from bus $i$ to bus $j$:
\begin{equation}\label{eq:Pij}
P_{ij} = V_i^2(g_{si}+g_{ij}) -V_iV_j(g_{ij}\cos \theta _{ij}+b_{ij}\sin \theta _{ij})
\end{equation}
\begin{equation}\label{eq:Qij}
Q_{ij} = -V_i^2(b_{si}+b_{ij}) -V_iV_j(g_{ij}\sin \theta _{ij}-b_{ij}\cos \theta _{ij})
\end{equation}
\item Line current flow magnitude from bus $i$ to bus $j$:
\begin{equation}\label{eq:Iij}
I_{ij} = \sqrt {P_{ij}^2+Q_{ij}^2}/V_i
\end{equation}
\end{enumerate}
These functions represent the set of functions $h_i(x)$ that relate the state
variables $V_i$ and $\theta_i$ to the measurements. The variables in the above
expressions are defined as

\begin{itemize}
\item $V_i$ is the voltage magnitude at bus $i$
\item $\theta_i$ is the phase angle at bus $i$
\item $\theta_{ij} = \theta_i-\theta_j$
\item $G_{ij}+jB_{ij}$ is the $ij$th element of the Y-bus matrix
\item $g_{ij}+jb_{ij}$ is the admittance of the series branch between bus $i$ and bus $j$
\item $g_{si}+jb_{sj}$ is the admittance of the shunt branch at bus $i$ 
\end{itemize}

\subsection*{The measurement Jacobian $H$:}

The Jacobian matrix $H$ can be written as
\begin{equation}
H  =\begin{bmatrix}
\frac{\partial P_{inj}}{\partial \theta} & \frac{\partial P_{inj}}{\partial V}\\ 
\frac{\partial P_{flow}}{\partial \theta} & \frac{\partial P_{flow}}{\partial V}\\ 
\frac{\partial Q_{inj}}{\partial \theta} & \frac{\partial Q_{inj}}{\partial V}\\ 
\frac{\partial Q_{flow}}{\partial \theta} & \frac{\partial Q_{flow}}{\partial V}\\ 
\frac{\partial I_{mag}}{\partial \theta} & \frac{\partial I_{mag}}{\partial V}\\ 
0 & \frac{\partial V_{mag}}{\partial V}
\end{bmatrix}
\end{equation}

where the expressions for each block are
\begin{equation}
\frac{\partial P_{i}}{\partial \theta_i}  =\sum _{j=1}^N V_iV_j(-G_{ij}\sin\theta_{ij}+B_{ij}\cos{\theta_{ij}})-V_i^2B_{ii}
\end{equation}
\begin{equation}
\frac{\partial P_{i}}{\partial \theta_j}  =V_iV_j(G_{ij}\sin\theta_{ij}-B_{ij}\cos{\theta_{ij}})  
\end{equation}
\begin{equation}
\frac{\partial P_{i}}{\partial V_i}  =\sum _{j=1}^N V_j(G_{ij}\cos\theta_{ij}+B_{ij}\sin{\theta_{ij}})+V_iG_{ii} 
\end{equation}
\begin{equation}
\frac{\partial P_{i}}{\partial V_j}  =V_i(G_{ij}\cos\theta_{ij}+B_{ij}\sin{\theta_{ij}})
\end{equation}
\begin{equation}
\frac{\partial Q_{i}}{\partial \theta_i}  =\sum _{j=1}^N V_iV_j(G_{ij}\cos\theta_{ij}+B_{ij}\sin{\theta_{ij}})-V_i^2G_{ii}
\end{equation}
\begin{equation}
\frac{\partial Q_{i}}{\partial \theta_j}  =V_iV_j(-G_{ij}\cos\theta_{ij}-B_{ij}\sin{\theta_{ij}})  
\end{equation}
\begin{equation}
\frac{\partial Q_{i}}{\partial V_i}  =\sum _{j=1}^N V_j(G_{ij}\sin\theta_{ij}-B_{ij}\cos{\theta_{ij}})-V_iB_{ii} 
\end{equation}
\begin{equation}
\frac{\partial Q_{i}}{\partial V_j}  =V_i(G_{ij}\sin\theta_{ij}-B_{ij}\cos{\theta_{ij}}) 
\end{equation}
\begin{equation}
\frac{\partial P_{ij}}{\partial \theta_i}  =V_iV_j(g_{ij}\sin\theta_{ij}-b_{ij}\cos{\theta_{ij}})
\end{equation}
\begin{equation}
\frac{\partial P_{ij}}{\partial \theta_j}  =-V_iV_j(g_{ij}\sin\theta_{ij}-b_{ij}\cos{\theta_{ij}})
\end{equation}
\begin{equation}
\frac{\partial P_{ij}}{\partial V_i}  =-V_j(g_{ij}\cos\theta_{ij}+b_{ij}\sin{\theta_{ij}})+2(g_{ij}+g_{si})V_i
\end{equation}
\begin{equation}
\frac{\partial P_{ij}}{\partial V_j}  =-V_i(g_{ij}\cos\theta_{ij}+b_{ij}\sin{\theta_{ij}})
\end{equation}
\begin{equation}
\frac{\partial Q_{ij}}{\partial \theta_i}  = -V_iV_j(g_{ij}\cos\theta_{ij} + b_{ij}\sin{\theta_{ij}})
\end{equation}
\begin{equation}
\frac{\partial Q_{ij}}{\partial \theta_j}  = V_iV_j(g_{ij}\cos\theta_{ij} + b_{ij}\sin{\theta_{ij}})
\end{equation}
\begin{equation}
\frac{\partial Q_{ij}}{\partial V_i}  =-V_j(g_{ij}\sin\theta_{ij}-b{ij}\cos{\theta_{ij}})-2(b_{ij}+b_{si})V_i
\end{equation}
\begin{equation}
\frac{\partial Q_{ij}}{\partial V_j}  =-V_i(g_{ij}\sin\theta_{ij}-b{ij}\cos{\theta_{ij}})
\end{equation}
\begin{equation}
\frac{\partial V_i}{\partial V_i}  = 1
\end{equation}
\begin{equation}
\frac{\partial V_i}{\partial V_j}  = 0
\end{equation}
\begin{equation}
\frac{\partial V_i}{\partial \theta_i}  = 0
\end{equation}
\begin{equation}
\frac{\partial V_i}{\partial \theta_j}  = 0
\end{equation}
\begin{equation}
\frac{\partial I_{ij}}{\partial \theta_i}  = \frac{g_{ij}^2+b_{ij}^2}{I_{ij}}V_iV_j\sin{\theta_{ij}} 
\end{equation}
\begin{equation}
\frac{\partial I_{ij}}{\partial \theta_j}  = -\frac{g_{ij}^2+b_{ij}^2}{I_{ij}}V_iV_j\sin{\theta_{ij}}
\end{equation}
\begin{equation}
\frac{\partial I_{ij}}{\partial V_i}  = \frac{g_{ij}^2+b_{ij}^2}{I_{ij}}(V_i-V_j\cos{\theta_{ij}})
\end{equation}
\begin{equation}
\frac{\partial I_{ij}}{\partial V_i}  = \frac{g_{ij}^2+b_{ij}^2}{I_{ij}}(V_j-V_i\cos{\theta_{ij}})
\end{equation}


\end{document}

