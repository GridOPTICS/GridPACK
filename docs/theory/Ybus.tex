\documentclass[12pt]{article}
\begin{document}
\title{YBUS Admittance Matrix Formulation}
\author{Yousu Chen \\PNNL}
\maketitle
This document is a description for how to formulate the YBUS admittance matrix. In General, the diagonal terms $Y_{ii}$ are the self admittance terms, equal to the sum of the admittances of all devices incident to bus $i$.  The off-diagonal terms $Y_{ij}$ are equal to the negative of the sum of the admittances joining the two buses. Shunt terms only affect the diagonal entries in the $Y$ matrix. For large systems, $Y$ is a sparse matrix and it is structually symmetric. 

\subsection*{For transmission lines}
\begin{equation}
Y_{ij} = \frac{-1}{r_{k}+j\*x_{k}}
\end{equation} 
\begin{equation}
Y_{ji} = Y_{ij}
\end{equation} 
\begin{equation}
Y_{ii} = - \sum\nolimits_{i\ne j} Y_{ij} 
\end{equation} 
\begin{equation}
Y_{jj} = - \sum\nolimits_{j\ne i} Y_{ji}
\end{equation} 
where:\\*
$Y_{ij}$: the ${ij}_{th}$ element in the YBus matrix.\\* 
$i$: the "from" bus.\\*
$j$: the "to" bus. \\*
$k$: the $k_{th}$ transmission line from $i$ to $j$. \\*
$r_{k}$: the resistance of the $k_{th}$ transmission line.\\*
$x_{k}$: the reactance of the $k_{th}$ transmission line.
\subsection*{For transformers}
For tap changing and phase shifting transformer, in general the off-nominal tap value can be considered as a complex numberi $a$, where the tap ratio is $t$ and the phase shifter is $\theta$.  Therefore, $a$ = $t*(\cos{\theta}+j*\sin{\theta})$. \\*
Let define
\begin{equation}
yt = \frac{-1}{r_{k}+j\*x_{k}}
\end{equation} 
\begin{equation}
Y_{ij} = -\frac{yt}{a^{*}}
\end{equation} 
\begin{equation}
Y_{ij} = -\frac{yt}{a}
\end{equation} 
\begin{equation}
Y_{ii} = \frac{yt}{{|a|}^2}
\end{equation} 
\begin{equation}
Y_{jj} = yt 
\end{equation} 
where:\\*
$Y_{ij}$: the ${ij}_{th}$ element in the $Y$ matrix.\\* 
$i$: the "from" bus.\\*
$j$: the "to" bus. \\*
$k$: the $k_{th}$ transformer from $i$ to $j$. \\*
$r_{k}$: the resistance of the $k_{th}$ transformer.\\*
$x_{k}$: the reactance of the $k_{th}$ transformer.\\*
$a^*$: the conjugate of $a$.\\*
\\*
Given the bus admittance matrix $Y$ for the entire system, the transformer model can be introduced by modifying the following 4 elements:
\begin{equation}
Y_{ij}^{new} = Y_{ij} - \frac{yt}{a^{*}}
\end{equation} 
\begin{equation}
Y_{ji}^{new} = Y_{ji} - \frac{yt}{a}
\end{equation} 
\begin{equation}
Y_{ii}^{new} = Y_{ii} + \frac{yt}{{|a|}^2}
\end{equation} 
\begin{equation}
Y_{jj}^{new} = Y_{jj} + yt
\end{equation} 
\subsection*{For shunts}
Shunts only contribute to diagonal elements. The sources of shunts include:
\begin{itemize}
\item shunt devices located at buses $(gs+j\*bs)$;
\item transmission line/transformer charging $b_{k}$ (distributed half to each end)
     from end: $b_{ki}$= $0.5\*b_{k}$;
     to end: $b_{kj}$= $0.5\*b_{k}$;
\item transmission line/transformer shunt admittance, which normally a small value: $(gi+j\*bi)$; 
\end{itemize}
Therefore, the general equation for diagonal elements is:
\begin{equation}
Y_{ii} = - (\sum\nolimits_{i\ne j} Y_{ij}) + j\*b_{ki} + gs_{i}+j\*bs_{i} + gi_{ki}+j\*bi_{ki}
\end{equation} 
\begin{equation}
Y_{jj} = - (\sum\nolimits_{j\ne i} Y_{ji}) + j\*b_{kj} + gs_{j}+j\*bs_{j} + gj_{kj}+j\*bj_{kj}
\end{equation} 
where:\\*
$Y_{ij}$: the ${ij}_{th}$ element in the YBus matrix.\\* 
$i$: the "from" bus. \\*
$j$: the "to" bus. \\*
$k$: the $k_{th}$ transmission line/transformer from $i$ to $j$. \\*
$gs_{i}+j*bs_{i}$: the shunt at bus $i$. \\*
$b_{k}$: the line charging of the $k_{th}$ line. \\*
$b_{ki}$ = 0.5 *$b_{k}$ : the line charging of the $k_{th}$ line assigned to "from" end $i$. \\*
$b_{kj}$ = 0.5 *$b_{k}$ : the line charging of the $k_{th}$ line assigned to "to" end $j$. \\*
$gi_{ki}+bi_{ki}$: the $k_{th}$ line shunt admitance at "from" end $i$. \\*
$gj_{kj}+bi_{kj}$: the $k_{th}$ line shunt admitance at "to" end $j$. \\*


\begin{center}
\textit{Thanks!}
\end{center}
\end{document}

